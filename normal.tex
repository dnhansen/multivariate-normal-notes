% Document setup
\documentclass[article, a4paper, 11pt, oneside]{memoir}
\usepackage[utf8]{inputenc}
\usepackage[T1]{fontenc}
\usepackage[UKenglish]{babel}

% Document info
\newcommand\doctitle{Multivariate normal distribution}
\newcommand\docauthor{Danny Nygård Hansen}

% Formatting and layout
\usepackage[autostyle]{csquotes}
\usepackage[final]{microtype}
\usepackage{xcolor}
\frenchspacing
\usepackage{articlepagestyle}
\usepackage{articlesectionstyle}

% Fonts
\usepackage[largesmallcaps]{kpfonts}
\DeclareSymbolFontAlphabet{\mathrm}{operators} % https://tex.stackexchange.com/questions/40874/kpfonts-siunitx-and-math-alphabets
\linespread{1.06}
\let\mathfrak\undefined
\usepackage{eufrak}
\usepackage{inconsolata}
\usepackage{amssymb}

% Hyperlinks
\usepackage{hyperref}
\definecolor{linkcolor}{HTML}{4f4fa3}
\hypersetup{%
	pdftitle=\doctitle,
	pdfauthor=\docauthor,
	colorlinks,
	linkcolor=linkcolor,
	citecolor=linkcolor,
	urlcolor=linkcolor,
	bookmarksnumbered=true
}

% Equation numbering
\numberwithin{equation}{chapter}

% Footnotes
\footmarkstyle{\textsuperscript{#1}\hspace{0.25em}}

% Mathematics
\usepackage{basicmathcommands}
\usepackage{framedtheorems}
\usepackage{probabilitycommands}

% Lists
\usepackage{enumitem}
\setenumerate[0]{label=\normalfont(\arabic*)}

% Title
\title{\doctitle}
\author{\docauthor}


\newcommand{\mat}{\mathrm{M}}
\newcommand{\trans}{\top}

\begin{document}

\maketitle



% \noindent Throughout we let $(\Omega, \calF, \P)$ denote a fixed probability space.

\begin{definition}[Multivariate normal distribution]
    An $n$-dimensional random vector $\rvar X$ follows an $n$-variate normal distribution if its characteristic function is given by
    %
    \begin{equation*}
        \phi_{\rvar X}(t)
            = \exp \Bigl( \iu \inner{t}{\mu} - \frac{1}{2} t^\trans \Sigma t \Bigr),
            \qquad t \in \reals^n,
    \end{equation*}
    %
    for some $\mu \in \reals^n$ and an $n \times n$ positive semidefinite matrix $\Sigma$. In this case we write $\rvar{X} \dist \normaldist[n]{\mu}{\Sigma}$.
\end{definition}
%
Recall that a $\Sigma$ is said to be positive semidefinite if $t^\trans \Sigma t \geq 0$ for all $t \in \reals^n$.




\begin{proposition}
    Let
        $\rvar X \dist \normaldist[n]{\mu}{\Sigma}$,
        $a \in \reals^n$ and
        $B \in \mat_{p \times n}(\reals)$.
    Then
        $a + B \rvar X \dist \normaldist[n]{a + B\mu}{B \Sigma B^\trans}$.
    In particular,
        $\inner{t}{\rvar X} \dist \normaldist{
            \inner{t}{\mu}
        }{
            t^\trans \Sigma t
        }$
    for all $t \in \reals^n$.

    Conversely, if $\rvar X$ is an $n$-dimensional random variable, and $\inner{t}{\rvar X}$ is normally distributed for all $t \in \reals^n$, then $\rvar X$ is $n$-variate normally distributed. In fact, if there exist $\mu \in \reals^n$ and $\Sigma \in \mat_n(\reals)$ such that
        $\inner{t}{\rvar X} \dist \normaldist{
            \inner{t}{\mu}
        }{
            t^\trans \Sigma t
        }$
    for all $t \in \reals^n$, then $\rvar X \dist \normaldist[n]{\mu}{\Sigma}$.
\end{proposition}

\begin{proof}
    First notice that $B \Sigma B^\trans$ is clearly positive semidefinite since $\Sigma$ is. Furthermore,
    %
    \begin{align*}
        \phi_{a+B \rvar X}(s)
            &= \e^{\iu \inner{s}{a}} \exp \Bigl( \iu \inner{B^\trans s}{\mu} - \frac{1}{2} (B^\trans s)^\trans \Sigma (B^\trans s) \Bigr) \\
            &= \exp \Bigl( \iu \inner{s}{a + B\mu} - \frac{1}{2} s^\trans (B \Sigma B^\trans) s \Bigr)
    \end{align*}
    %
    for all $s \in \reals^n$ as desired.

    For the converse direction, first assume that $\inner{t}{\rvar X}$ is normally distributed for all $t \in \reals^n$. In particular, the coordinates $\rvar X_1, \ldots, \rvar X_n$ are normally distributed, and hence $\rvar X$ has a well-defined mean vector $\mu$ and covariance matrix $\Sigma$. Furthermore, we have $\mean{\inner{t}{\rvar X}} = \inner{t}{\mu}$ and $\var{\inner{t}{\rvar X}} = t^\trans \Sigma t$. (See Andersen.) It follows that
    %
    \begin{equation*}
        \phi_{\rvar X}(t)
            = \phi_{\inner{t}{\rvar X}}(1)
            = \exp \Bigl( \iu \inner{t}{\mu} - \frac{1}{2} t^\trans \Sigma t \Bigr)
    \end{equation*}
    %
    for $t \in \reals^n$. The final claim follows similarly.
\end{proof}


\begin{corollary}
    If $\rvar X \dist \normaldist[n]{\mu}{\Sigma}$, then $\mu$ is the mean vector and $\Sigma$ the covariance matrix for $\rvar X$.
\end{corollary}

\begin{remark}
    Let $\mu \in \reals^n$, and let $\Sigma \in \mat_n(\reals)$ be positive semidefinite. We show that there exists a random variable $\rvar X$ with distribution $\normaldist[n]{\mu}{\Sigma}$.

    First let $\Sigma^{1/2} \in \mat_n(\reals)$ be the (unique) positive semidefinite matrix such that $\Sigma = \Sigma^{1/2} \Sigma^{1/2}$. We might construct $\Sigma^{1/2}$ as follows: Since $\Sigma$ is symmetric there exists a diagonal matrix $D$ and an orthogonal matrix $Q$ such that $\Sigma = Q^\trans DQ$. Since $\Sigma$ is positive semidefinite, the entries in $D$ are non-negative. Call the diagonal entries $\lambda_1, \ldots, \lambda_n$. Let $D^{1/2}$ be the diagonal matrix whose entries along the diagonal are $\sqrt{\lambda_1}, \ldots, \sqrt{\lambda_n}$, and let $\Sigma^{1/2} = Q^\trans D^{1/2} Q$.

    Now let $\rvar U_1, \ldots \rvar U_n$ be i.i.d. $N(0,1)$-distributed variables, and let $\rvar U = (\rvar U_1, \ldots, \rvar U_n)$. For $t = (t_1, \ldots, t_n) \in \reals^n$, the characteristic function of $\rvar U$ is then
    %
    \begin{equation*}
        \phi_{\rvar U}(t)
            = \bigprod_{j=1}^n \phi_{\rvar U_j}(t_j)
            = \bigprod_{j=1}^n \exp \Bigl( - \frac{1}{2} t_j^2 \Bigr)
            = \exp \Bigl( - \frac{1}{2} t^\trans I t \Bigr),
    \end{equation*}
    %
    where $I$ is the identity matrix. It follows that $\rvar U \dist \normaldist[n]{0}{I}$. Letting $\rvar X = \mu + \Sigma^{1/2} \rvar U$, we find that $\rvar X \dist \normaldist[n]{\mu}{\Sigma}$.
\end{remark}

\end{document}